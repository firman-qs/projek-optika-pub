\section{Kesimpulan}
Kesimpulan dari percobaan yang dilakukan adalah,
\begin{enumerate}[leftmargin=*]
    \item Hubungan antara waktu pengukuran terhadap hasil ukur intensitas cahaya matahari menggunakan Lux Meter adalah berbanding lurus untuk kelompok waktu pagi ke siang, berbanding terbalik untuk kelompok waktu siang ke malam, dan nol pada waktu malam hingga dini hari.
    \item Posisi matahari, dalam hal ini altitude, berhubungan dengan intensitas cahaya matahari yang terkur dengan tren berbanding lurus.
    \item Suhu udara berbanding lurus dengan intensitas cahaya matahari yang datang.
    \item Kelembapan udara berbanding terbalik dengan intensitas cahaya matahari yang datang.
\end{enumerate}

\section{Saran}
Selanjutnya dapat dilakukan kajian yang lebih spesifik dan tidak terbatas hanya pada hubungan antara intensitas cahaya matahari dengan posisi, suhu udara, dan kelembapan udara tiap waktunya secara umum. Sebelum melakukan percobaan maka peneliti perlu memahami dengan baik variabel-variabel apa yang akan diukur dan variabel apa yang pengaruhnya akan dikontrol. Dalam hal alat ukur, dapat digunakan alat ukur khusus untuk pengukuran intensitas cahaya matahari, sehingga hasil pengukuran tidak lagi bergantung terhadap kualitas sensor cahaya dari \textit{smartphone} yang memang telah disesuaikan untuk kebutuhan tertentu.
