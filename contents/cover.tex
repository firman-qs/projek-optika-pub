\clearpage
\hspace*{-0.26in}
\begin{minipage}{0.3\textwidth}
\flushleft\includegraphics[width=2cm]{gambar/Lambang-UM.png}
\end{minipage}
%% temporary titles
% command to provide stretchy vertical space in proportion
\newcommand\nbvspace[1][3]{\vspace*{\stretch{#1}}}
% allow some slack to avoid under/overfull boxes
\newcommand\nbstretchyspace{\spaceskip0.5em plus 0.25em minus 0.25em}
% To improve spacing on titlepages
\newcommand{\nbtitlestretch}{\spaceskip0.3em}
\pagestyle{empty}
\begin{flushleft}
\bfseries
% \nbvspace[1]
\vspace{6mm}
\Huge
{
	\nbtitlestretch\fontsize{18pt}{18pt}\selectfont PENGUKURAN INTENSITAS CAHAYA MATAHARI DALAM SATU HARI DENGAN MENGGUNAKAN LUX METER
}

\vspace{0.5em}
\normalfont\small Projek I OPTIKA
\vspace{1em}

\normalfont\small Disusun oleh:\\
\normalfont\small Firman Qashdus Sabil (210321606892), Offering: AC\\%[0.5em]
\normalfont\small untuk Memenuhi Tugas Mata Kuliah Optika \\
\normalfont\small yang diampu Dr. Cahyo Aji Hapsoro M.Si

\vspace{15em}\begin{center}
\begin{tikzpicture}[overlay, remember picture, node distance={2pt}, transform canvas={scale=0.9}]
	\node (tapir) {\includegraphics[width=10cm]{gambar/cover_hero.jpg}};
	% \node[black!85, above right=of tapir.north, rotate=5] (fundcalc) {$E(\si{lux})=\frac{\phi(\si{lumen})}{A(\si{m^2})}$};
	% \node[black!70, above left=of fundcalc, rotate=-5] (orth) {$d\vB = \frac{\mu_0 I}{4\pi} \frac{d\vl \times \hr}{r^2}$};
	% \node[black!75, above right=of orth, rotate=-3, xshift=15pt] (diff) {$\mathbf{\nabla}\cdot\mathbf{B}=0$};
	% \node[black!60, right=of diff, rotate=3, yshift=-15pt] (vecdef) {$\mathbf{B}=\frac{\mu_0 I}{2 \pi R}$};
	% \node[black!40, below left=of orth, rotate=-9] (euler) {$\mathbf{\nabla}\times\mathbf{B}=\mu_0 \mathbf{J}+\frac{1}{c^2}\frac{\partial \mathbf{E}}{\partial t}$};
	% \node[black!30, above left=of diff] (eigendecomp) {$\sin{\phi}$};
	% \node[black!90, below right=of fundcalc, xshift=10pt, yshift=15pt, rotate=-10] (taylorcos) {$\bar{b}=\frac{n \sum\left(x y\right)-\sum x \sum y}{n \sum x^2-\left(\sum x\right)^2}$};
	% \node[black!25, above right=of diff, xshift=10pt, yshift=-10pt] (gamma) {$r=\frac{s}{\cos{\alpha}}$};
	% \node[black!20, above=of diff, xshift=-5pt, yshift=15pt, rotate=-7] (lintrans) {$S_a = S_y \sqrt{\frac{\sum x^2}{n\sum x^2 - (\sum x)^2}}\ \text{  dan  }\ S_b = S_y \sqrt{\frac{n}{n\sum x^2 - (\sum x)^2}}$};
	% \node[black!20, above left=of eigendecomp, rotate=-7] (binom) {$R_a =\frac{S_a}{a}$};
	% \node[black!15, above left=of orth, rotate=3] (rot) {$\bar{a}=\frac{\left(\sum y\right)\left(\sum x^2\right)-\left(\sum x\right)\left(\sum x y\right)}{n\left(\sum x^2\right)-\left(\sum x\right)^2}$};
	% \node[black!13, above left=of gamma, yshift=15pt, rotate=5] (expand) {$(a+b)^{n}=\sum\limits_{k=0}^{n}\binom{n}{k}a^{n-k}b^{k}$};
	% \node[black!10, above left=of binom, rotate=3] (elog) {$a^{b}=\eu^{b\log(a)}$};
\end{tikzpicture}\end{center}
\nbvspace[3]
\normalsize

\large

\vspace{2em}
\fontsize{13pt}{13pt}\bfseries\sffamily\selectfont UNIVERSITAS NEGERI MALANG\\\vspace{0.2cm}
\fontsize{13pt}{13pt}\bfseries\sffamily\selectfont FAKULTAS MATEMATIKA DAN ILMU PENGETAHUAN ALAM\\\vspace{0.2cm}
\fontsize{13pt}{13pt}\bfseries\sffamily\selectfont PROGRAM STUDI PENDIDIKAN FISIKA\\\vspace{0.2cm}
\fontsize{13pt}{13pt}\bfseries\sffamily\selectfont DESEMBER 2023




\end{flushleft}
\newpage
\pagestyle{fancy} % otherwise fancy headers disappear
