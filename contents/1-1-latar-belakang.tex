Matahari, sebagai bintang pusat tata surya, menyediakan sumber energi utama bagi kehidupan di Bumi. Dari Bumi, matahari berjarak sekitar 150 juta kilometer. Energi matahari diterima oleh Bumi dalam bentuk radiasi elektromagnetik \cite{blal2020prediction}. Radiasi elektromagnetik ini mencakup berbagai panjang gelombang, termasuk cahaya tampak \cite{rokhaniyah2019alat}. Cahaya matahari adalah pendorong utama semua proses atmosfer, iklim, serta merupakan kunci bagi kelangsungan hidup ekosistem di Bumi \cite{hader2007effects,zarnetske2021potential}. Seiring perputaran Bumi pada sumbunya, terjadi perubahan siklus siang dan malam yang menciptakan variasi harian dalam intensitas cahaya matahari. 

Intensitas cahaya matahari yang bervariasi dapat mempengaruhi suhu udara di berbagai lokasi geografis di Bumi. Suhu udara dapat terpengaruh karena seluruh energi dari matahari, sampai ke Bumi dalam bentuk radiasi matahari \cite{wald:hal-01676634}. Radiasi ini merupakan salah satu cara untuk mendistribusikan panas. Inilah yang mengakibatkan intensitas cahaya matahari berpengaruh terhadap suhu udara yang dilewatinya. Energi panas dari sinar matahari diserap oleh partikel-partikel udara, benda, dan permukaan bumi \cite{Lipinski2021-lp}. Pemanasan yang disebabkan oleh paparan langsung matahari dapat meningkatkan suhu di siang hari, sementara kurangnya radiasi matahari pada malam hari dapat menyebabkan penurunan suhu. Selain suhu, intensitas cahaya matahari juga mempengaruhi kelembapan udara. Proses penguapan yang ditingkatkan oleh sinar matahari dapat meningkatkan kelembapan udara di sekitarnya. 

Oleh karena itu, dilakukan percobaan pengukuran intensitas cahaya matahari tiap jam selama 24 jam. Untuk mengukur intensitas cahaya matahari tiap jamnya digunakan perangkat \textit{smartphone} yang memiliki sensor cahaya dengan aplikasi \textit{Pengukur Cahaya}. Besarnya intensitas cahaya matahari perlu diukur karen pada dasarnya segala aktivitas dan teknologi energi terbarukan banyak memanfaatkan matahari sebai sumber energi utamanya.