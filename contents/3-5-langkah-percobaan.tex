\subsection{Penentuan Lokasi Pengukuran}
\begin{enumerate}[leftmargin=*, label=\alph*]
    \item Mencari titik lokasi terbuka dimana tidak ada penghalang yang dapat mengganggu proses pengambilan data.
    \item Mendapatkan koordinat dari titik lokasi yang telah dipilih menggunakan \textit{Google Maps}.
    \item Menyimpan data lokasi pengukuran pada perangkat \textit{smartphone}.
\end{enumerate}
\subsection{Pengukuran Intensitas Cahaya Matahari}
\begin{enumerate}[leftmargin=*, label=\alph*]
    \item Menjalankan aplikasi \textit{Pengukur Cahaya} pada \textit{smartphone}.
    \item Memulai Pengambilan data pada menit ke nol tiap jamnya.
    \item Pengambilan data dilakukan sebanyak tiga kali.
    \item Menjalankan tangkapan layar pada alat pengukur (\textit{smartphone}) untuk data ke-1, ke-2, dan ke-3.
\end{enumerate}
\subsection{Penghitungan Posisi Matahari}
\begin{enumerate}[leftmargin=*, label=\alph*]
    \item Membuka aplikasi web \href{https://www.suncalc.org/#/-8.0905,112.6468,17/2023.11.10/18:00/1/3}{suncalc.org},
    \begin{figure}[H]
        \centering
        \includegraphics[width=0.5\textwidth]{gambar/Langkah/suncalc1.png}
        \caption{Hasil pencarian aplikasi \textit{suncalc} pada laman pencarian \textit{web browser}}
        \labfig{suncalc1}
    \end{figure}
    \item Mengatur tanggal dimana posisi matahari akan dihitung.
    \begin{figure}[H]
        \centering
        \includegraphics[width=0.5\textwidth]{gambar/Langkah/suncalc2.png}
        \caption{Hasil pencarian aplikasi \textit{suncalc} pada laman pencarian \textit{web browser}}
        \labfig{suncalc2}
    \end{figure}
    \item Menghitung posisi matahari tiap jam pada menit ke nol dengan menggeser \textit{slider} waktu pada bagian atas.
    \begin{figure}[H]
        \centering
        \includegraphics[width=0.5\textwidth]{gambar/Langkah/suncalc4.png}
        \caption{Hasil pencarian aplikasi \textit{suncalc} pada laman pencarian \textit{web browser}}
        \labfig{suncalc3}
    \end{figure}
    \item Data yang didapat berupa altitude, azimuth, dan deklinasi dari matahari.
    % \begin{figure}[H]
    %     \centering
    %     \includegraphics[width=0.5\textwidth]{gambar/Langkah/suncalc4.png}
    %     \caption{Hasil pencarian aplikasi \textit{suncalc} pada laman pencarian \textit{web browser}}
    %     \labfig{suncalc4}
    % \end{figure}
\end{enumerate}
\subsection{Pengambilan Data Cuaca di Lokasi Pengukuran}
\begin{enumerate}[leftmargin=*, label=\alph*]
    \item Membuka aplikasi web \href{https://www.visualcrossing.com/weather-history}{visualcrossing}.
    \begin{figure}[H]
        \centering
        \includegraphics[width=0.5\textwidth]{gambar/Langkah/weather1.png}
        \caption{Halaman awal web \href{https://www.visualcrossing.com/weather-history}{visualcrossing}}
        \labfig{weather1}
    \end{figure}
    \item Mendaftar akun agar dapat melakukan pengunduhan data cuaca.
    \begin{figure}[H]
        \centering
        \includegraphics[width=0.5\textwidth]{gambar/Langkah/weather2.png}
        \caption{Pendaftaran akun \href{https://www.visualcrossing.com/weather-history}{visualcrossing}}
        \labfig{weather2}
    \end{figure}
    \item Melakukan pencarian data cuaca sesusai lokasi pengukuran intensitas cahaya matahari.
    \begin{figure}[H]
        \centering
        \includegraphics[width=0.5\textwidth]{gambar/Langkah/weather3.png}
        \caption{Penacarian cuaca pada \href{https://www.visualcrossing.com/weather-history}{visualcrossing}}
        \labfig{weather3}
    \end{figure}
    \item Mengatur tanggal data cuaca sesuai tanggal pengukuran intensitas cahaya matahari.
    \begin{figure}[H]
        \centering
        \includegraphics[width=0.5\textwidth]{gambar/Langkah/weather4.png}
        \caption{Pengaturan tanggal data cuaca \href{https://www.visualcrossing.com/weather-history}{visualcrossing}}
        \labfig{weather4}
    \end{figure}
    \item Melakukan pengunduhan data cuaca. Data cuaca berbentuk dokumen \textit{comma separated value} (csv).
    \begin{figure}[H]
        \centering
        \includegraphics[width=0.5\textwidth]{gambar/Langkah/weather5.png}
        \caption{Pengaturan tanggal data cuaca \href{https://www.visualcrossing.com/weather-history}{visualcrossing}}
        \labfig{weather5}
    \end{figure}
\end{enumerate}
\subsection{Dokumentasi Kegiatan}
Setelah melakukan pengukuran intensitas cahaya dilakukan foto dokumentasi sebagai bukti pengukuran di lapangan.