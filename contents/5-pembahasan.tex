\section{Hubungan Posisi Matahari Terhadap Hasil Ukur Intensitas Cahaya Matahari Tiap Waktu}
Berdasarkan analisis data dapat diketahui bahwa terdapat hubungan antara waktu pengambilan data/pengukuran terhadap hasil ukur intensitas cahaya matahari yang didapat. Cahaya matahari mulai tampak pada pukul 05.00 WIB dan intensitasnya terus meningkat hingga pukul 11.00 WIB. Setelah pukul 11.00 WIB intensitas cahaya matahari mulai menurun hingga titik terkecilnya pada pukul 17.00 WIB. Pada pukul 18.00 s.d. 04.00 tidak ada intensitas cahaya matahari yang dapat diukur.  Ketiga bagian waktu ini dapat kita kelompokkan kedalam kelompok pagi-siang (pukul 05.00 s.d. 11.00 WIB), siang-sore (pukul 11.00 s.d. 17.00), dan malam-pagi (pukul 18.00 s.d. 04.00). Perlu diingat bahwa pengukuran dimulai pada tanggal 10 November pukul 19.00 dan berakhir pada 11 November pukul 18.00, tetapi disini tanggal atau hari tidak begitu penting. Hal yang penting adalah bahwa data diambil secara berkesinambungan selama 24 jam. 

Pada kelompok pagi-siang, intensitas cahaya matahari mulai terukur dan meningkat. Hal ini karena posisi matahari telah berada pada altitude $\geq 0^\circ$, sehingga setidaknya sebagian dari matahari sudah tampak di ufuk/horizon dan pandaran cahayanya dapat terdeteksi oleh \textit{Pengukur Cahaya/Lux meter}. Selanjutnya intensitas cahaya matahari terus bertambah sampai titik pengukuran maksimum pada pukul 11.00 dengan intensitas $153.539,6$  Lux. Pada pukul 11.00 posisi matahari berada pada altitude $80,27^\circ$ yang mana adalah altitude maksimum dalam projek ini. Secara teori, tentu, puncak intensitas cahaya matahari akan terukur antara pukul 11.00 dan 12.00, saat altitude nya $90^\circ$.

Pada kelompok siang-sore, intensitas cahaya matahari menurun sampai dengan intensitas terendahnya pada pukul 17.00 WIB dengan intensitas 1.727 Lux. Penurunan intensitas cahaya matahari ini terjadi seiring dengan menurunnya altitude matahari. Pada pukul 17.00 WIB, di mana matahari terbenam, altitude matahari berada pada $5,54^\circ$. Pada pukul 17.00 WIB altitude matahari lebih besar dari $0^\circ$ sehingga jika dibandingkan dengan intensitas pada pagi hari pukul 5.00 WIB didapatkan hasil pengukuran yang lebih besar.

Pada kelompok malam-pagi, intensitas cahaya matahari tidak dapat diukur sama sekali. Hal ini mengakibatkan pembacaan Lux meter menjadi 0 Lux. Pada kelompok malam-pagi posisi matahari berada pada altitude negatif. Altitude negatif berarti bahwa matahari berada di bawah horizon pada waktu tertentu ditinjau dari suatu tempat, sehingga tidak ada intensitas cahaya matahari yang dapat terukur. Pada saat ini matahari mulai berpindah ke belahan lain dari Bumi sebagai akibat dari rotasi bumi sambil melakukan revolusi terhadap matahari. 

Dalam pembahasan pada tiga pengelompokan tersebut, perlu diingat bahwa hubungan antara posisi dan intensitas cahaya matahari tiap waktunya tidak dapat dimodelkan kedalam kesebandingan atau ketidaksebandingan linier. Di sini kita baru membahas, posisi matahari ditinjau dari altitude atau sudut ketinggiannya dari posisi pengukur. Terdapat dua data lain yang menjelaskan posisi matahari, yakni sudut azimuth dan deklinasi. Ketiga data posisi (altitude, azimuth, dan deklinasi) ini dapat dianalisis lebih lanjut untuk menentukan lintasan matahari pada tiap waktu, hari, bulan, maupun tahun. Namun, hal ini akan melibatkan perhitungan dan tinjauan geometri yang kompleks.

Selain posisi, sebenarnya tedapat faktor-faktor lain yang dapat mempengaruhi hasil pengukuran intensitas cahaya matahari. Cahaya matahari yang melewati atmosfer dengan berbagai macam partikel akan mengalami hamburan dan serapan. Serapann dan hamburan, seperti hamburan Rayleigh, Mie, dan non selektif mempengaruhi rentang panjang gelombang yang sampai pada pengukur. Panjang gelombang cahaya yang datang kepada pengukur tentu mempengaruhi pembacaan intensitas terukur. Selain itu jika terjadi hamburan non selektif akibat sekumpulan awan hujan yang tebal, tentu akan mengurangi intensitas cahaya matahari yang terukur oleh Lux Meter.

\section{Hubungan Intensitas Cahaya Matahari Terhadap Suhu Udara}
Pada bagian ini data juga diurutkan mulai dari intensitas cahaya matahari terkecil, yakni 0 Lux sampai dengan intensitas cahaya terbesar 153.539,6 Lux. Untuk menjaga kontinuitas data, untuk intensitas cahaya matahari terkecil (0 Lux), dari data pukul 18.00 s.d. 04.00, diambil data pada pukul 04.00 WIB. Sedangkang untuk intensitas cahaya matahari terbesar terdapat pada data pukul 11.00 WIB. Data intensitas cahaya matahari dan suhu udara tiap waktu ini kemudian di plot dalam garfik sebaran dengan analisis regresi linier.

Berdaskan analisis data, suhu udara menunjukkan tren yang berbanding lurus dengan intensitas cahaya matahari. Suhu udara minimum terukur sebesar $19,3^\circ$ pada intensitas cahaya matahari 309,3 Lux. Data suhu minimum ini terukur pada pukul 05.00 WIB. Sedangkan suhu udara maksimum yakni $28,0^\circ$ pada intensitas cahaya matahari 117.049,3 Lux. Data suhu maksimum ini terukur pada pukul 10.00. 

Seluruh energi dari matahari, sampai ke Bumi dalam bentuk radiasi matahari, sebagai bagian dari sekumpulan energi yang disebut spektrum radiasi elektromagnetik. Radiasi matahari memuat cahaya tampak, cahaya ultraviolet, infrared, radio, sinar-X, dan sinar Gamma. Radiasi ini merupakan salah satu cara untuk mendistribusikan panas. Inilah yang mengakibatkan intensitas cahaya matahari berpengaruh terhadap suhu udara yang dilewatinya.

\section{Hubungan Intensitas Cahaya Matahari Terhadap Kelembapan Udara}
Pada bagian ini data juga diurutkan mulai dari intensitas cahaya matahari terkecil, yakni 0 Lux sampai dengan intensitas cahaya terbesar 153.539,6 Lux. Untuk menjaga kontinuitas data, untuk intensitas cahaya matahari terkecil (0 Lux), dari data pukul 18.00 s.d. 04.00, diambil data pada pukul 04.00 WIB. Sedangkang untuk intensitas cahaya matahari terbesar terdapat pada data pukul 11.00 WIB. Data intensitas cahaya matahari dan kelembapan udara tiap waktu ini kemudian di plot dalam garfik sebaran dengan analisis regresi linier.

Suhu udara, tentu berbanding terbalik dengan kelembapan udara. Seiring dengan pertambahan panas udara maka udara akan mengering karena jumlah uap air yang berkurang. Dari sini dapat dikatahui bahwa hubungan antara intensitas cahaya matahari yang sebanding dengan suhu pasti akan berbanding terbalik dengan kelembapan udara. Dapat dilihat pada bagian analisis data, nilai minimum kelembapan udara, yakni 56,66\% pada intensitas cahaya matahari 117.049,3 Lux. Sedangkan kelembapan udara maksimum sebesar 98,15\% pada intensitas cahaya matahari 309,3 Lux.