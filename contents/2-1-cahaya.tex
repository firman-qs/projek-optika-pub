Sebelum permulaan abad ke-19, cahaya dianggap sebagai sebuah aliran partikel yang dipancarkan oleh suatu sumber yang dilihat oleh mata \cite{Serway2018-if}. Newton sebagai arsitek utama model partikel dari cahaya, berpendapat bahwa partikel dipancarkan dari sumber cahaya dan partikel ini merangsang indra penglihatan saat memasuki mata \cite{Serway2018-if}. Dengan menggunakan ide ini, dia mampu menjelaskan pemantulan dan pembiasan. Sebaliknya Christian Huygens menunjukkan bahwa model gelombang dari cahaya juga dapat menjelaskan pemantulan dan pembiasan \cite{Pedrotti2006-ek}. Sifat cahaya sebagai gelombang diperkuat dengan eksperimen interferensi cahaya yang dilakukan Thomas Young dan hasil kerja Maxwell yang menunjukkan bahwa cahaya adalah bentuk dari gelombang elektromagnet dengan frekuensi tinggi \cite{Hecht2015-by}.

Model cahaya sebagai gelombang dan teori elektromagnet klasik mampu menjelaskan berbagai sifat cahaya, tetapi terdapat beberapa hasil eksperimen yang tidak dapat dijelaskan dengan model ini, misalnya efek fotolistrik \cite{Hecht2015-by}. Einstein mengusulkan penjelasan dari efek fotolistrik menggunakan pemodelan berdasarkan kuantisasi yang dikembangkan Max Planck \cite{Krane2012-fm}. Model kuantisasi cahaya ini menyatakan bahwa energi dari sebuah gelombang cahaya ada dalam sebuah partikel foton \cite{Krane2012-fm}. Cahaya merambat dalam ruang vakum dengan kecepatan $3\times 10^8\ \si{m/s}$

\section{Cahaya Matahari}
\subsection{Radiasi Cahaya Matahari}
Radiasi matahari merupakan gelombang elektromagnetik dengan komponen medan listrik $\vec{E}$ dan medan magnet $\vec{B}$ \cite{Hecht2015-by}. Radiasi cahaya matahari yang mengenai Bumi disebut insolasi. Intensitas radiasi sinar matahari berkurang sebesar $1/R^2$ jarak dari matahari.
\subsection{Intensitas Cahaya Matahari}
Intensitas adalah besaran daya yang dipancarkan oleh cahaya persatuan luas. Satuan dari intensitas cahaya adalah Candela (CD). Alat yang digunakan untuk mengukur intensitas cahaya adalah luxmeter. 

\section{Pengukuran Cahaya Matahari}
\subsection{Istilah dan Besaran yang Diukur Pada Pengukuran Cahaya Matahari}
Beberapa istilah yang perlu diketahui dalam pengukuran intensitas cahaya matahari adalah
\begin{description}
    \item[Luminous Flux] berasal dari bahasa latin \q{Fluxus} yang artinya aliran. Dalam hal cahaya, luminous flux diartikan sebagai arus cahaya $\Pi$ yang dipancarkan tiap waktu. Luminous Flux memiliki satuan Lumen(lm).
    \item[Illuminan] Adalah banyaknya luminous flux tiap satuan luas. Illuminan memiliki satuan Lux
    \item[Intensitas Cahaya] adalah besarnya cahaya tampak yang dipancarkan dalam satuan waktu untuk tiap sudut ruang. Satuan dari intensitas cahaya adalah Candela.
\end{description}

\subsection{Sudut Datang atau Altitude dari Sinar Matahari}
Sudut datang dari sinar matahari adalah sudut yang terbentuk antara sinar datang matahari terhadap permukaan bumi. Saat matahari tepat berada di suatu titik (membentuk sudut datang $90^\circ$) maka penyinaran matahari terjadi secara maksimal sehingga suhunya juga maksimal. Sedangkan pada sudut datang kurang atau lebih dari $90^\circ$ maka penyinaran matahari semakin berkurang. Semakin besar sudut datang sinar matahari maka sinar tersebut akan terproyeksi pada satuan yang lebih sempit, sehingga energi tiap satuan luas yang diterima semakin besar. Selain itu, semakin besar sudut datang sinar matahari maka semakin pendek lintasan atmosfer yang dilalui sinar \cite{herreria2020solar}. Penggambaran sudut altitude dapat dilihat pada \reffig{azimuthx}.
\subsection{Letak Lintang dari Titik Pengukuran Cahaya Matahari}
Bentuk bulat dari Bumi mengakibatkan variasi perolehan cahaya matahari di tempat yang berbeda. Salah satu garis pembagi bumi adalah garis lintang. Garis lintang adalah garis imajiner yang membentang dari timur ke barat di sepanjang permukaan Bumi dan diukur dalam derajat terhadap garis khatulistiwa. Garis lintang diukur dari $0^\circ$ (khatulistiwa) hingga maksimal $90^\circ$ di utaran atau selatan. Daerah dengan lintang $0^\circ$ lebih dekat dengan matahari sehingga menerima cahaya matahari lebih banyk dibandingkan dengan daerah pada lintang $90^\circ$. Dalam percobaan ini, titik pengukuran terletak pada garis lintang $-8,0904651^\circ$.

\subsection{Sudut Azimuth Matahari dari Titik Pengukuran Cahaya Matahari}
Sudut azimuth seperti kompas yang menunjukkan arah terbitnya matahari. Pada siang hari untuk belahan Bumi bagian utara matahari selalu tepat berada di sebelah selatan dan sebaliknya untuk belahan bumi bagian selatan. Penggambaran sudut azimuth dapat dilihat pada \reffig{azimuthx}.
\begin{figure}[H]
    \centering
    \includegraphics[width=0.4\textwidth]{gambar/azimuth.png}
    \caption{Sudut altitude dan azimuth. Sudut azimuth berperan sebagai arah kompas dengan utara $0^\circ$ dan selatan $180^\circ$}
    \labfig{azimuthx}
\end{figure}

\subsection{Sudut Deklinasi Matahari}
Sudut deklinasi yang disimbolkan dengan $\delta$ bervariasi tiap musimnya karena kemiringan bumi pada sumbu rotasinya dan rotasi bumi mengelilingi matahari. Kemiringan bumi terhadap sumbu rotasinya adalah sebesar $\pm 23,45^\circ$
\begin{figure}[H]
    \centering
    \includegraphics[width=0.8\textwidth]{gambar/declination1.png}
    \caption{Sudut deklinasi Bumi tiap musim }
    \labfig{declination 1}
\end{figure}
Deklinasi matahari adalah sudut antara garis khatulistiwa dan garis yang ditarik dari pusat bumi ke pusat matahari.
\begin{figure}[H]
    \centering
    \includegraphics[width=0.8\textwidth]{gambar/declination2.png}
    \caption{Sudut deklinasi matahari}
    \labfig{declination 2}
\end{figure}
Sudut deklinasi dapat dihitung dengan persamaan
\begin{equation*}
    \delta = -23,45^\circ \times \cos\rbrak{\frac{360}{365}\times \rbrak{d+10}},
\end{equation*}
dimana $d$ adalah hari dalam satu tahun, dengan 1 Januari sebagai $d=1$.
% \subsection{Kejernihan Atmosfer}
% text

\section{Hubungan Suhu dengan Kelembapan}
Kemampuan udara untuk mempertahankan uap air bergantung pada suhunya. Dengan adanya perubahan suhu udara, kemampuan mempertahankan kelembapan dapat meningkat atau menurun, sehingga memengaruhi kelembapan relatif. Hubungan antara kelembaban dan suhu berbanding terbalik. Jika suhu meningkat, maka kelembaban relatif akan berkurang, dengan demikian udara akan menjadi lebih kering. Saat suhu menurun, udara akan menjadi lebih basah, oleh karena itu kelembaban relatif akan meningkat \cite{widhiada2019robust}.

\section{Instrumen Pengukuran Intensitas Cahaya Matahari Berbasis Android}
Selama beberapa tahun terakhir, \textit{smartphone} berbasis Android atau iPhone dianggap sebagai ponsel multifungsi yang canggih. Dalam kasus Android, biasanya terdapat setidaknya empat sensor berbeda yang memungkinkan pengguna memantau beberapa besaran seperti kelembapan, pencahayaan, tekanan, dan suhu sekitar. Sensor cahaya misalnya, digunakan oleh \textit{smartphone} untuk mengatur pencahayaan layar secara otomatis \cite{esmaeili2019mobile}. Sebagai contoh pada ruangan yang terang, maka kecerahan layar \textit{smartphone} akan meningkat agar dapat tetap dibaca oleh pengguna dangan baik. Sebaliknya pada ruangan yang gelap kecerahan layar \textit{smartphone} akan menurun agar mata pengguna tidak kelelahan. Sensor cahaya pada smartphone mampu menghasilkan pengukuran yang akurat. Namun, secara teoritis, sensor tersebut tidak dapat dibandingkan dengan keakuratan perangkat keras khusus yang memang di desain untuk mengukur intensitas cahaya matahari.

Dalam rangka menghubungkan \textit{hardware} sensor cahaya dengan antarmuka pengguna maka diperlukan sebuah \textit{software} pengukur cahaya. Pada sistem operasi Android, terdapat banyak pilihan aplikasi sensor cahaya yang dapat digunakan. Pada projek ini nama aplikasi yang digunakan adalah \textit{Pengukur Caaya} sebagaimana pada bagian langkah percobaan.

\section{Database Cuaca}
Database cuaca disini maksudnya adalah media-media yang menyimpan histori cuaca berupa temperatur, prediksi hujan, kelembapan, dan parameter-parameter cuaca lainnya. Salah satu penyedia histori cuaca adalah \href{https://www.visualcrossing.com/about}{visualcrossing.com}. Platform ini menyediakan histori laporan cuaca dalam bentuk data cuaca dan \textit{application programming interface} (API).
\begin{figure}[H]
    \centering
    \includegraphics[width=0.8\textwidth]{gambar/Langkah/vc1.png}
    \caption{Platform penyedia histori laporan cuaca \href{https://www.visualcrossing.com/about}{visualcrossing.com}}
\end{figure}

\section{Instrumen Perhitungan Posisi Matahari Berbasis Web}
Aplikasi web \href{https://www.suncalc.org/#/-8.0905,112.6468,18/2023.11.10/18:00/1/3}{suncalc} digunakan untuk melakukan perhitungan posisi matahari. Data posisi matahari yang bisa didapat dari aplikasi web ini berupa data altitude, azimuth, deklinasi, dan beberapa data lain. Pengguna dapat menghitung data posisi matari berdasarkan tanggal dan jam yang ditentukan.
\begin{figure}[H]
    \centering
    \includegraphics[width=0.8\textwidth]{gambar/Langkah/vc2.png}
    \caption{Aplikasi perhitungan posisi matahari \href{https://www.suncalc.org}{suncalc}}
\end{figure}

\section{Instrumen Penentuan Letak di Permukaan Bumi (GPS)}
Google maps adalah aplikasi yang dapat digunakan untuk menentukan titik pengambilan data. Untuk dapat menggunakannya, \textit{smartphone} harus memiliki akses internet dan sistem GPS yang aktif.
\begin{figure}[H]
    \centering
    \includegraphics[width=0.8\textwidth]{gambar/Langkah/maps.png}
    \caption{Aplikasi perhitungan posisi pengukuran \href{https://www.google.com/maps}{Google maps}}
\end{figure}
