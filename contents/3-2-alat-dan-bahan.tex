Alat dan bahan yang digunakan dalam percobaan ini adalah
\begin{center}
        \setlength\LTleft{-2mm}\begin{longtable}{p{0.2cm}  p{6cm}  p{6cm}}  
            % \toprule
            %%%%%%%% Next ROw %%%%%%%% 
            \text{1.}
            &
            \adjustbox{valign=t}{\includegraphics[width=6cm]{gambar/Dokumentasi/smartphone-complete.pdf}}\vspace{0.5cm}
            &
            \textit{Smartphone} Oppo A17 dengan sensor cahaya dibagian depan, aplikasi jam bawaan, dan aplikasi \textit{Pengukur Cahaya}.\\
             %%%%%%%% Next ROw %%%%%%%% 
            % \cmidrule(r){1-3}
            \text{2.}
            &
            \adjustbox{valign=t}{\includegraphics[width=6cm]{gambar/Dokumentasi/jam-digital-com.pdf}}\vspace{0.5cm}
            &
            Jam digital dari Gawai.\\
            %%%%%%%% Next ROw %%%%%%%% 
            % \cmidrule(r){1-3}
            \text{3.}
            &
            \adjustbox{valign=t}{\includegraphics[width=6cm]{gambar/Dokumentasi/lux-meter-com.pdf}}\vspace{0.5cm}
            &
            Aplikasi \textit{Pengukur Cahaya} untuk mengukur intensitas cahaya matahari.\\
            %%%%%%%% Next ROw %%%%%%%% 
            % \cmidrule(r){1-3}
            \text{4.}
            &
            \adjustbox{valign=t}{\includegraphics[width=6cm]{gambar/Dokumentasi/suncalc-comp.png}}\vspace{0.5cm}
            &
            Aplikasi web \href{https://www.suncalc.org/#/-8.0905,112.6468,17/2023.11.10/18:00/1/3}{suncalc.org} untuk menghitung posisi matahari (derajat azimuth, altitude, dan deklinasi) dari lokasi pengukuran.\\
            % \bottomrule
            %%%%%%%% Next ROw %%%%%%%% 
            % \cmidrule(r){1-3}
            \text{5.}
            &
            \adjustbox{valign=t}{\includegraphics[width=6cm]{gambar/Dokumentasi/visualcrossing.png}}\vspace{0.5cm}
            &
            Aplikasi web histori cuaca \href{https://www.visualcrossing.com/weather-history}{visualcrossing} untuk mendapatkan kondisi cuaca dari lokasi pengukuran.\\
            % \bottomrule
        \end{longtable}
    \label{table:alat-dan-bahan}
\end{center}
\vspace{-3em}